\documentclass{article}
\usepackage[utf8]{inputenc}
\usepackage[icelandic]{babel}
\usepackage[T1]{fontenc}
\usepackage{graphicx}
\usepackage{mathtools}
\usepackage{amsmath}
\usepackage{amssymb}
\usepackage{minted}


\graphicspath{ {./imgs} }
\title{Snake - Viðmótsforritun}
\author{ttb3@hi.is}
\date{\today}


\begin{document}
\maketitle


\section*{Gagnvirkniskröfur notenda}
\subsection*{1.}
\subsubsection*{Gagnvirkni}
    Notandi getur spilað Snake þar sem snákurinn getur borðað mat,stækkað og liðast, 
    snákurinn deyr svo ef hann snertir eitursnák eða klessir á sinn eiginn líkama.
\subsubsection*{Lýsing á verkefni} 
    Í fyrri útgáfu af Snake var aðeins hægt að lengjast en ekki liðast. 
    Með viðbættri liðunarhegðun þarf að breyta stjórnun á snáknum, 
    athuga þarf í hvaða átt snákurinn stefnir og koma í veg fyrir að hann snúi við.
\subsubsection*{Staða kröfu}
    Krafa komin

\subsection*{2.}
\subsubsection*{Gagnvirkni}
Notandi getur skráð stigin sín niður ásamt þremur einkennisstöfum,
eins og í gömlum spilakössum. Þessi stig eiga að geymast á milli leikja.

\subsubsection*{Lýsing á verkefni}
Hingað til hefur stigafjöldi verið vistaður sjálfkrafa og með enga vísun í notenda.
stigin voru aðeins til staðar frá því að leikurinn hafðist og þangað til að honum var lokað,
ekkert var geymt á milli keyrslna.
\subsubsection*{Staða kröfu}
    Krafa komin

\subsection*{3.}
\subsubsection*{Gagnvirkni}
Notandi getur hefnt sín á eitursnákum, öðru hvoru er maturinn sem birtist 'sérstakur'
á þann hátt að þegar notandi borðar sérstaka matinn, 
getur hann ekki dáið og ef snákurinn rekst á eitursnák deyr eitursnákurinn.
Mjög svipuð pæling og stóru doppurnar í pac-man og stjörnur í super mario.

\subsubsection*{Lýsing á verkefni}
Eins og er í leiknum eru eitursnákarnir bara til staðar til að angra og hægja á leiknum.
Eftir nokkrar mínútur af leik voru komnir svo margir eitursnákar að varla var hægt að spila lengur.
Með þessari viðbót ætti að vera hægt að spila lengur, passa þarf bara að gefa ekki of mikið af sérstökum mat.
\subsubsection*{Staða kröfu}
    Útfærsla tókst ekki vegna þess að tími var ekki nægilegur

\subsection*{4.}
\subsubsection*{Gagnvirkni}
Notandi getur pásað leik með mús, space eða escape. 
Þegar notandi pásar leikinn fær viðkomandi yfirlit yfir hversu langur tími er liðinn,
hversu mikinn mat er búið borða, hvar hann er í röðinni með stig á þessum tímapunkti, 
hversu margir eitursnákar hafa birst, hversu margir eitursnákar hafa dáið og hversu mikið af sérstökum mat er búið að borða.

\subsubsection*{Lýsing á verkefni}
Í augnablikinu er bara hægt að pása með músarsmelli, því þarf að breyta í esc og space. 
Þegar notandi pásar núna gerist ekkert nema að leikurinn frýs. Bætist við öll gögnin um leikinn sem eru talin upp í gegnvirknispartinum.

\subsubsection*{Staða kröfu}
    Krafa hálf komin, sleppti því að bæta við space og tók músina út því ég var alltaf að ýta á hana óvart

\subsection*{5.}
\subsubsection*{Gagnvirkni}
Startskjár, 
bæta við almennilegum startskjá þar sem hægt er að fá yfirlit yfir stig 
og hugsanlega einhver auka skemmtileg gögn.

\subsubsection*{Lýsing á verkefni}
Leikurinn eins og er notast við startskjá sem er illa gerður, 
eina sem hægt er að sjá er smá texti og eina sem hægt er að gera er að byrja leikinn.
Bæta við stigalista og betra útliti.

\subsubsection*{Staða kröfu}
    Krafa komin

\section*{Útlits- og hönnunarkröfur}
Leikurinn eins og hann lítur út núna er forljótur.
\subsection*{1.}
\subsubsection*{Útlitskröfur}
Allir leikjahlutir eiga að notast við sprites frekar en einföld form gerð í javafx.
Það þarf ekki að vera sjúklegur munur á þessum spritum, 
eitur og player snákar geta litið eins út nema með mismunandi litum.

\subsubsection*{Lýsing á kröfu}
Núna eru snákarnir bara kassar með útlínur og maturinn rauðir hringir, mjög ljótt.
Með þessari kröfu mun leikurinn byrja að líta út eins og eitthvað sem er ekki sett saman af smábarni.

\subsubsection*{Staða kröfu}
krafa komin

\end{document}