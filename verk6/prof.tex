\documentclass{article}
\usepackage[utf8]{inputenc}
\usepackage[icelandic]{babel}
\usepackage[T1]{fontenc}
\usepackage{graphicx}
\usepackage{mathtools}
\usepackage{amsmath}
\usepackage{amssymb}
\usepackage{minted}


\graphicspath{ {./} }
\title{Notendapróf - Spil21}
\author{ttb3@hi.is}
\date{\today}

\begin{document}
\maketitle

\section{Forritið}
\begin{tabular}{|l|l|}
    \hline
    Nafn forrits & Spil21\\
    Höfundur forrits & Sverrir Sigfússon\\
    Dagsetning útgáfu forrits & 11. apríl 2022\\
    Umsjónamaður prófana& Þorvaldur Tumi Baldursson\\
    \hline
\end{tabular}

\section{Verkefnin}
\begin{tabular}{|l|l|l|l|}
    \hline
    númer verkefnis &Texti verkefnis og gögn                                &Númer kröfu    &Tegund kröfu\\
    \hline
    1               &Breyttu um tungumál                                    &2              &Ú\\
    2               &Spilaðu leik, fara aftur í menu og byrja nýjan         &4              &G\\
    3               &Skoðaðu reglurnar                                      &1              &Ú\\
    4               &Breyttu um litaþema                                    &3              &Ú\\
    5               &Hlustaðu á hljóðinn, slökktu á hljóðinu og hlustaðu    &3              &G\\
    \hline
\end{tabular}
Athugasemdir:
\\Þegar notandi klárar leik og reynir að komast aftur í menu er aðeins einn takki sem gæti komið til greina og hann lokar forritinu.

\section{Framkvæmd prófana}
\begin{tabular}{|l|l|l|l|}
    \hline
    Nafn þáttakanda         &Í viðmótsforritun?     &Dagsetning     &Staðsetning prófanna\\
    \hline
    Kjartan Óli Ágústsson   &Já                     &11. apríl      &Nördakjallarinn\\
    \hline
    Gunnar Björn Þrastarson &Já                     &11. apríl      &Nördakjallarinn\\
    \hline
\end{tabular}

\section{Niðurstöður prófana}
\subsection*{Heildarlisti yfir villur/vandamál}
\begin{tabular}{p{0.1\textwidth}p{0.4\textwidth}p{0.1\textwidth}p{0.1\textwidth}p{0.1\textwidth}}
    \hline
    Númer villu &Stutt lýsing&nr. verkefnis&Nöfn notenda&Alvarleiki\\
    \hline
    2           &Notendur lokuðu forritinu þegar þeir reyndu að komast aftur í menu eftir að spila leik&4G&Gunnar og Kjartan&L\\ 
    \hline          
\end{tabular}

\end{document}